% !TeX root = ./0-Tesis-de-maestria-JLBG.tex
Desarrollando las Ecs. \eqref{eq: Ly spectral density r theta}, \eqref{eq: Lx spectral density r theta} y \eqref{eq: Lz spectral density r theta} se obtiene
%
\begin{align}
\mathcal{L}_x &= \frac{R^3}{\pi}\int_0^{\pi}\int_0^{2\pi} \sin\theta \Var{\sin\varphi \,\mathcal{T}_{\theta r} + \cos\theta\cos\varphi\,\mathcal{T}_{\varphi r}} \,d\varphi \, d\theta, \label{eq: Lx spectral integral angles}\\
\mathcal{L}_y &= \frac{R^3}{\pi}\int_0^{\pi}\int_0^{2\pi} \sin\theta \Var{\cos\varphi \,\mathcal{T}_{\theta r} - \cos\theta\sin\varphi\,\mathcal{T}_{\varphi r}} \,d\varphi \, d\theta, \label{eq: Ly spectral integral angles}\\
\mathcal{L}_z &= \frac{R^3}{\pi}\int_0^{\pi}\int_0^{2\pi} \sin^{2}\theta \, \mathcal{T}_{\varphi r} \,d\varphi \, d\theta, \label{eq: Lz spectral integral angles}
\end{align}
% 
donde 
\begin{align}
\mathcal{T}_{\theta r} &= {\rm Re}\Var{\frac{\epsilon_0}{\cgs{4\pi\epsilon_0}} E_{\theta}\varsw E_{r}^{*}\varsw + \frac{\mu_0}{\cgs{4\pi\mu_0}} H_{\theta}\varsw H_{r}^{*}\varsw  },\\
\mathcal{T}_{\varphi r} &= {\rm Re}\Var{\frac{\epsilon_0}{\cgs{4\pi\epsilon_0}} E_{\varphi}\varsw E_{r}^{*}\varsw + \frac{\mu_0}{\cgs{4\pi\mu_0}} H_{\varphi}\varsw H_{r}^{*}\varsw  }.
\end{align}

Se puede realizar la separación $E_i=E_i^{\rm e}+E_i^{\rm s}$, para desarrollar el término $E_{\theta}\varsw E_{r}^{*}\varsw$ en fución de las sumas multipolar expresada en la Ecs. \eqref{Eq: multipolar ext E field} y \eqref{Eq: multipolar scat E field} de la siguiente manera
\begin{equation}
E^{\rm a}_{\alpha}\varsw E_{r}^{\rm b \,*}\varsw = \sum_{\ell, m} \sum_{\ell^{\prime}, m^{\prime}} \mathscr{E}_{\ell, m}^{\text{a}\alpha}\mathscr{E}_{\ell^{\prime}, m^{\prime}}^{\text{br}\,*},\label{eq: cross terms E field}
\end{equation}
donde $a$ y $b$ pueden tomar los valores de $s$ (denotando componentes de campo esparcido) o de $e$ (componentes de campo externo), y donde $\alpha$ puede denotar la proyección en $\hat{ \theta}$ o en $\hat{\varphi}$. 

Se procederá a desarrollar cada una de las componentes representadas por la Ec. \eqref{eq: cross terms E field}, así como sus análogas para el campo $\vv{H}$. Posteriormente se integrarán cada una de las expresiones obtenidas de manera analítica, para obtener una expresión exacta para la contribución espectral de la transferencia de momento angular de cada multipolo, mediante las Ecs. \eqref{eq: Lx spectral integral angles}, \eqref{eq: Ly spectral integral angles} y \eqref{eq: Lz spectral integral angles}.

El término $\mathscr{E}_{\ell, m}^{\text{a}\theta}\mathscr{E}_{\ell^{\prime}, m^{\prime}}^{\text{br}\,*}$ se puede escribir como
\begin{align}
\mathscr{E}_{\ell, m}^{\text{a}\theta}\mathscr{E}_{\ell^{\prime}, m^{\prime}}^{\text{br}\,*}=& \, \Bigg\{-C_{\ell,m}^{\rm a}D_{\ell^{\prime}, m^{\prime}}^{\rm b \, *}\frac{Z_{\ell}^{\rm a}Z_{\ell^{\prime}}^{\rm b\, *}}{k_0 r} m \frac{P_{\ell}^{m}P_{\ell^{\prime}}^{m^{\prime}}}{\sin\theta}-D_{\ell, m}^{\rm a}D_{\ell^{\prime}, m^{\prime}}^{\rm b \, *} \frac{f_{\ell}^{\rm a}Z_{\ell^{\prime}}^{\rm b \, *}}{k_0 r}\nonumber\\
& \times \Var{(\ell+1) \frac{P_{\ell}^m P_{\ell^{\prime}}^{m^{\prime}}\cos\theta}{\sin\theta}-(l-m+1)\frac{P_{\ell+1}^{m}P_{\ell^{\prime}}^{m^{\prime}}}{\sin\theta}}\Bigg\}\ell^{\prime}(\ell^{\prime}+1)  \rme^{\rmi (m-m^{\prime})\varphi },
\end{align}
donde $Z_{\ell}^{\rm s}=h_{\ell}^{+}(k_o r)$ para el campo esparcido, y $Z_{\ell}^{\rm e}=j_{\ell}(k_o r)$ para el campo externo, y se ha definido:
\begin{equation}
f_{\ell}^{\rm a}= \var{\ell + 1} \frac{Z_{\ell}^{\rm a}}{k_0 r}-Z_{\ell+1}^{\rm a}.
\end{equation}
Análogamente, el término $\mathscr{E}_{\ell, m}^{\text{a}\varphi}\mathscr{E}_{\ell^{\prime}, m^{\prime}}^{\text{br}\,*}$ se puede escribir como
\begin{align}
\mathscr{E}_{\ell, m}^{\text{a}\varphi}\mathscr{E}_{\ell^{\prime}, m^{\prime}}^{\text{br}\,*} =& \, \Bigg\{ C_{\ell, m}^{\rm a} D_{\ell^{\prime}, m^{\prime}}^{\rm b \, *} \frac{Z_{\ell}^{\rm a} Z_{\ell^{\prime}}^{\rm b \, *}}{k_0 r} \Var{\var{\ell +1} \frac{P_{\ell}^{m} P_{\ell^{\prime}}^{m^{\prime}}\cos\theta}{\sin\theta} -\var{\ell - m +1}\frac{P_{\ell+1}^m P_{\ell^{\prime}}}{\sin\theta} }\nonumber\\
& + D_{\ell, m}^{\rm a} D_{\ell^{\prime}, m^{\prime}}^{\rm b \, *} m \frac{f_{\ell}^{\rm a} Z_{\ell^{\prime}}^{\rm b\, *}}{k_0 r} \frac{P_{\ell}^{m} P_{\ell^{\prime}}^{m^{\prime}}}{\sin\theta} \Bigg\} \, \rmi \, \ell^{\prime}(\ell^{\prime}+1)  \rme^{\rmi (m-m^{\prime})\varphi }
\end{align}

Es útil calcular primero la integral en $\varphi$, teniendo en cuenta que 
\begin{align}
\int_0^{2\pi} \rme^{\rmi (m-m^{\prime}) \varphi} \cos\varphi  \, d\varphi &= \pi \var{ \delta_{m+1, m^{\prime}} + \delta_{m-1, m^{\prime}} },\\
\int_0^{2\pi} \rme^{\rmi (m-m^{\prime}) \varphi} \sin\varphi  \, d\varphi &= \rmi \pi \var{ \delta_{m-1, m^{\prime}} - \delta_{m+1, m^{\prime}} },
\end{align}
y para calcular la integral en $\theta$ se definen las siguientes cantidades
\begin{align}
IM_{\ell,\ell^{\prime}}^{m,m^{\prime}} &=\int_{-1}^1 P_{\ell}^{m}(x) P_{\ell^{\prime}}^{m^{\prime}}(x) \,x \, dx, \\
IU_{\ell,\ell^{\prime}}^{m,m^{\prime}} &=\int_{-1}^1 P_{\ell}^{m}(x) P_{\ell^{\prime}}^{m^{\prime}}(x) \, \frac{dx}{\sqrt{1-x^2}}, \\
IV_{\ell,\ell^{\prime}}^{m,m^{\prime}} &=\int_{-1}^1 P_{\ell}^{m}(x) P_{\ell^{\prime}}^{m^{\prime}}(x) \, \frac{x^2}{\sqrt{1-x^2}} \, dx, \\
IW_{\ell,\ell^{\prime}}^{m,m^{\prime}} &=\int_{-1}^1 P_{\ell}^{m}(x) P_{\ell^{\prime}}^{m^{\prime}}(x) \, \frac{x}{\sqrt{1-x^2}} \, dx, 
\end{align}
que pueden ser calculadas mediante cuadraturas Gaussianas \cite{kahaner1989numerical}, que dan resultados exactos al ser utilizadas para integrar polinomios\footnote{Debe tenerse en cuenta que solo las funciones asociadas de Legendre $P_{\ell}^m$ de orden $m$ par son polinomios, por lo que solo las integrales pares en $m$ se calcularán de forma exacta.}. Se debe tomar en cuenta la relación de ortogonalidad de las funciones asociadas de Legendre \cite{Abramowitz}
\begin{equation}
\int_{-1}^{1} P_{\ell}^{m}(x) P_{\ell^{\prime}}^{m}(x) \, dx = \Delta_{\ell \ell^{\prime}} \qquad \text{con} \qquad \Delta_{\ell \ell^{\prime}}= \frac{2(\ell+m)!}{(2\ell+1)(\ell-m)!} \delta_{\ell \ell^{\prime}}.
\end{equation}
Con esto en mente se puede proceder a calcular las siguientes integrales
\begin{align}
\int_{0}^{4\pi} \mathscr{E}_{\ell, m}^{\text{a}\theta}\mathscr{E}_{\ell^{\prime}, m^{\prime}}^{\text{br}\,*} \sin\varphi \, d\Omega =& 
%
\, \rmi \pi \var{ \delta_{m-1, m^{\prime}} - \delta_{m+1, m^{\prime}} } \Bigg\{-C_{\ell,m}^{\rm a}D_{\ell^{\prime}, m^{\prime}}^{\rm b \, *}\frac{Z_{\ell}^{\rm a}Z_{\ell^{\prime}}^{\rm b\, *}}{k_0 R} m IU_{\ell, \ell^{\prime}}^{m, m^{\prime}} 
%
-D_{\ell, m}^{\rm a}D_{\ell^{\prime}, m^{\prime}}^{\rm b \, *} \frac{f_{\ell}^{\rm a}Z_{\ell^{\prime}}^{\rm b \, *}}{k_0 R} \nonumber \\
%
& \times \Bigg[ (\ell+1) IW_{\ell, \ell^{\prime}}^{m, m^{\prime}} 
%
-(l-m+1) IU_{\ell + 1, \ell^{\prime}}^{m, m^{\prime}} \Bigg] \Bigg\}\ell^{\prime}(\ell^{\prime}+1),\\
%
%
%
\int_{0}^{4\pi} \mathscr{E}_{\ell, m}^{\text{a}\theta}\mathscr{E}_{\ell^{\prime}, m^{\prime}}^{\text{br}\,*} \cos\varphi \, d\Omega =& 
%
\, \pi \var{ \delta_{m-1, m^{\prime}} + \delta_{m+1, m^{\prime}} } \Bigg\{-C_{\ell,m}^{\rm a}D_{\ell^{\prime}, m^{\prime}}^{\rm b \, *}\frac{Z_{\ell}^{\rm a}Z_{\ell^{\prime}}^{\rm b\, *}}{k_0 R} m IU_{\ell, \ell^{\prime}}^{m, m^{\prime}} 
%
-D_{\ell, m}^{\rm a}D_{\ell^{\prime}, m^{\prime}}^{\rm b \, *} \frac{f_{\ell}^{\rm a}Z_{\ell^{\prime}}^{\rm b \, *}}{k_0 R} \nonumber \\
%
& \times \Bigg[ (\ell+1) IW_{\ell, \ell^{\prime}}^{m, m^{\prime}} 
%
-(l-m+1) IU_{\ell + 1, \ell^{\prime}}^{m, m^{\prime}} \Bigg] \Bigg\}\ell^{\prime}(\ell^{\prime}+1),\\
%
%
%
\int_0^{4\pi}\mathscr{E}_{\ell, m}^{\text{a}\varphi}\mathscr{E}_{\ell^{\prime}, m^{\prime}}^{\text{br}\,*}\cos\theta\sin\varphi \, d\Omega
%
 =& \, - \pi \var{ \delta_{m-1, m^{\prime}} - \delta_{m+1, m^{\prime}} } \Bigg\{ C_{\ell, m}^{\rm a} D_{\ell^{\prime}, m^{\prime}}^{\rm b \, *} \frac{Z_{\ell}^{\rm a} Z_{\ell^{\prime}}^{\rm b \, *}}{k_0 R} \Bigg[\var{\ell +1} IV_{\ell, \ell^{\prime}}^{m, m^{\prime}} \nonumber \\ 
% 
&-\var{\ell - m +1} IW_{\ell + 1, \ell^{\prime}}^{m, m^{\prime}} \Bigg]
%
 + D_{\ell, m}^{\rm a} D_{\ell^{\prime}, m^{\prime}}^{\rm b \, *} m \frac{f_{\ell}^{\rm a} Z_{\ell^{\prime}}^{\rm b\, *}}{k_0 R} IW_{\ell, \ell^{\prime}}^{m, m^{\prime}} \Bigg\} \, \ell^{\prime}(\ell^{\prime}+1),  \\
%
%
%
\int_0^{4\pi}\mathscr{E}_{\ell, m}^{\text{a}\varphi}\mathscr{E}_{\ell^{\prime}, m^{\prime}}^{\text{br}\,*}\cos\theta\cos\varphi \, d\Omega
%
 =& \, \rmi \, \pi \var{ \delta_{m-1, m^{\prime}} + \delta_{m+1, m^{\prime}} } \Bigg\{ C_{\ell, m}^{\rm a} D_{\ell^{\prime}, m^{\prime}}^{\rm b \, *} \frac{Z_{\ell}^{\rm a} Z_{\ell^{\prime}}^{\rm b \, *}}{k_0 R} \Bigg[\var{\ell +1} IV_{\ell, \ell^{\prime}}^{m, m^{\prime}} \nonumber \\ 
% 
&-\var{\ell - m +1} IW_{\ell + 1, \ell^{\prime}}^{m, m^{\prime}} \Bigg]
%
 + D_{\ell, m}^{\rm a} D_{\ell^{\prime}, m^{\prime}}^{\rm b \, *} m \frac{f_{\ell}^{\rm a} Z_{\ell^{\prime}}^{\rm b\, *}}{k_0 R} IW_{\ell, \ell^{\prime}}^{m, m^{\prime}} \, \ell^{\prime}(\ell^{\prime}+1),\\
%
%
%
\int_0^{4\pi}\mathscr{E}_{\ell, m}^{\text{a}\varphi}\mathscr{E}_{\ell^{\prime}, m^{\prime}}^{\text{br}\,*}\sin\theta \, d\Omega
%
 =& \, \rmi \, 2\pi \, \delta_{m m^{\prime}} \Bigg\{ C_{\ell, m}^{\rm a} D_{\ell^{\prime}, m^{\prime}}^{\rm b \, *} \frac{Z_{\ell}^{\rm a} Z_{\ell^{\prime}}^{\rm b \, *}}{k_0 R} \Bigg[\var{\ell +1} IM_{\ell, \ell^{\prime}}^{m, m^{\prime}} \nonumber \\ 
% 
&-\var{\ell - m +1} \Delta_{\ell + 1, \ell^{\prime}}  \Bigg]
%
+ D_{\ell, m}^{\rm a} D_{\ell^{\prime}, m^{\prime}}^{\rm b \, *} m \frac{f_{\ell}^{\rm a} Z_{\ell^{\prime}}^{\rm b\, *}}{k_0 R} \Delta_{\ell \ell^{\prime}} \Bigg\} \, \ell^{\prime}(\ell^{\prime}+1),
\end{align}
y definiendo las siguientes cantidades
\begin{align}
I\mathscr{E}S_{\ell m \ell^{\prime} m^{\prime}}^{\rm a \theta b r} 
=& \, \frac{R^3}{\pi} \int_{0}^{4\pi} \mathscr{E}_{\ell, m}^{\text{a}\theta}\mathscr{E}_{\ell^{\prime}, m^{\prime}}^{\text{br}\,*} \sin\varphi \, d\Omega ,
\\
%
%
%
I\mathscr{E}C_{\ell m \ell^{\prime} m^{\prime}}^{\rm a \theta b r} 
=& \, \frac{R^3}{\pi} \int_{0}^{4\pi} \mathscr{E}_{\ell, m}^{\text{a}\theta}\mathscr{E}_{\ell^{\prime}, m^{\prime}}^{\text{br}\,*} \cos\varphi \, d\Omega,
\\
%
%
%
I\mathscr{E}CS_{\ell m \ell^{\prime} m^{\prime}}^{\rm a \varphi b r}
=& \, \frac{R^3}{\pi} \int_0^{4\pi}\mathscr{E}_{\ell, m}^{\text{a}\varphi}\mathscr{E}_{\ell^{\prime}, m^{\prime}}^{\text{br}\,*}\cos\theta\sin\varphi \, d\Omega,  \\
%
%
%
I\mathscr{E}CC_{\ell m \ell^{\prime} m^{\prime}}^{\rm a \varphi b r}
 =& \, \frac{R^3}{\pi} \int_0^{4\pi}\mathscr{E}_{\ell, m}^{\text{a}\varphi}\mathscr{E}_{\ell^{\prime}, m^{\prime}}^{\text{br}\,*}\cos\theta\cos\varphi \, d\Omega, \\
%
%
%
I\mathscr{E}S_{\ell m \ell^{\prime} m^{\prime}}^{\rm a \varphi b r}
 =& \, \frac{R^3}{\pi} \int_0^{4\pi}\mathscr{E}_{\ell, m}^{\text{a}\varphi}\mathscr{E}_{\ell^{\prime}, m^{\prime}}^{\text{br}\,*}\sin\theta \, d\Omega ,
\end{align}
se pueden reescribir las Ecs. \eqref{eq: Lx spectral integral angles}, \eqref{eq: Ly spectral integral angles} y \eqref{eq: Lz spectral integral angles} como 
%
\begin{align}
\mathcal{L}_x &= 
\sum_{\ell, m} \sum_{\ell^{\prime}, m^{\prime}} 
\var{I\mathscr{E}S_{\ell m \ell^{\prime} m^{\prime}}^{\rm a \theta b r} + 
I\mathscr{E}CC_{\ell m \ell^{\prime} m^{\prime}}^{\rm a \varphi b r}}, \label{eq: Lx spectral integral angles solved}\\
% 
\mathcal{L}_y &= 
\sum_{\ell, m} \sum_{\ell^{\prime}, m^{\prime}}
\var{I\mathscr{E}C_{\ell m \ell^{\prime} m^{\prime}}^{\rm a \theta b r} + 
I\mathscr{E}CS_{\ell m \ell^{\prime} m^{\prime}}^{\rm a \varphi b r}}, \label{eq: Ly spectral integral angles solved}\\ 
%
\mathcal{L}_z &= 
\sum_{\ell, m} \sum_{\ell^{\prime}, m^{\prime}}
I\mathscr{E}S_{\ell m \ell^{\prime} m^{\prime}}^{\rm a \varphi b r}. \label{eq: Lz spectral integral angles solved}
\end{align}
%