% !TeX root = ./0-Tesis-de-maestria-JLBG.tex

Desde mediados del siglo pasado, se han propuesto diversos métodos para desarrollar técnicas de manipulación de objetos en la escala micro y nanométrica \cite{ashkin1970acceleration, ashkin1987optical, Ashkin, custance2009atomic, dholakia2011shaping, marago2013optical, romo2020controlled}. Las fuerzas electromagnéticas producidos por haces de luz enfocados, han conducido al desarrollo de las pinzas ópticas, que han demostrado ser de utilidad para atrapar y mover microobjetos, incluidos virus y bacterias, teniendo impacto en el desarrollo tecnológico y médico \cite{ashkin1970acceleration, ashkin1987optical, Ashkin}.  

En 2004, Javier García de Abajo publicó un trabajo donde aborda la posibilidad de manipular nanoobjetos mediante microscopios electrónicos de transmisión (TEMs por sus siglas en inglés) \cite{GarciadeAbajo0}. Posteriormente se observó experimentalmente que los TEM pueden ser usados para inducir movimiento y hacer rotar a nanopartículas (NPs) \cite{Batson01, zheng2012electron}, lo cual ha motivado al desarrollo de una nueva técnica de manipulación llamada <<pinzas electrónicas>> \cite{Batson, oleshko2005electron, Oleshko}, en analogía con las pinzas ópticas. En diversos estudios experimentales sobre pinzas electrónicas, se ha observado que la transferencia de momento angular y lineal del haz de electrones a la NP, depende tanto de la velocidad con la que viaja el haz de electrones así como del parámetro de impacto, que es la distancia efectiva entre la trayectoria del haz de electrones y el centro de la NP \cite{OLESHKO2013203, Oleshko, Batson, Batson01, zheng2012electron, xu2010transmission}.  Se ha mostrado que al modificar el parámetro de impacto, la interacción entre el haz de electrones y la NP puede ser atractiva o repulsiva, y también se ha reportado que es posible modificar la dirección del giro inducido sobre la NP \cite{OLESHKO2013203, Batson, Oleshko}. 

Para el desarrollo de la técnica de pinzas electrónicas, es necesario un entendimiento teórico del problema. La interacción de haces de electrones con nanopartículas esféricas ha sido abordado desde la perspectiva de la electrodinámica clásica \cite{GarciadeAbajo0, PRBCoronado, Lagos2, Batson2, xu2010transmission}, mediante la solución de las ecuaciones de Maxwell en el espacio de frecuencias. Los trabajos citados anteriormente se han centrado en el cálculo de transferencia de momento lineal, mediante el cálculo de una integral de superficie cerrada alrededor de la NP y una integral en el espacio de frecuencias. En estos trabajos, los campos electromagnéticos producidos por el electrón y la NP, se pueden expresar en una base esférica en términos de una expansión multipolar, lo que permite separar la contribución eléctrica y magnética de la interacción, así como también separar la contribución de cada orden multipolar. A pesar de que estos trabajos han logrado reproducir el comportamiento atractivo y repulsivo de la interacción, estudios recientes han mostrado que dichos trabajos obtuvieron resultados no físicos al modelar la respuesta electromagnética de la NP mediante funciones dieléctricas no causales \cite{castrejon2021effects, castrejon2021phdthesis}. En los trabajos citados anteriormente, se muestra que no aparece la interacción repulsiva reportada experimentalmente, si se elimina el comportamiento no causal de las funciones dieléctricas. La Ref. \cite{castrejon2021phdthesis} resuelve de forma semi-analítica las integrales en el espacio de frecuencia, que previamente se resolvían de forma numérica en las Refs. \cite{GarciadeAbajo0, PRBCoronado, Lagos2, Batson2, xu2010transmission}. Lo anterior permite conocer de manera exacta la contribución en el espacio de frecuencias de cada multipolo a la transferencia de momento lineal, logrando así calcular la transferencia de momento lineal, de electrones rápidos a NPs grandes (de hasta $a=50$ nm de radio). El cálculo de la transferencia de momento angular (TMA) solo había sido discutido brevemente en trabajos posteriores a 2021 (ver por ejemplo la Ref. \cite{GarciadeAbajo-1}). Sin embargo, recientemente se han publicado estudios de la TMA de electrones rápidos a NPs pequeñas, de hasta $a=5$ nm de radio \cite{castellanos2021phdthesis, castellanos2021angular,castellanos2023theory}, donde se resuelven numéricamente las integrales en el espacio de frecuencias. Resolver numéricamente la integral de superficie para la TMA puede representar un problema en cuanto a los tiempos de cómputo, al calcular la TMA para NPs grandes. Una comparativa entre el avance logrado entre el caso lineal con el angular, sugiere que es necesario encontrar soluciones semi-analíticas en el espacio de frecuencias para disminuir el tiempo de cómputo en el cálculo de la TMA, y así poder estudiar la dinámica angular en la interacción de electrones rápidos con NPs grandes de hasta $a=50$ nm de radio.

Los haces electrónicos en un STEM pueden alcanzar los $400$ keV de energía cinética, con una corriente eléctrica del orden de pA.