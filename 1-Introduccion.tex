% !TeX root = ./0-Tesis-de-maestria-JLBG.tex

Desde mediados del siglo pasado, se han propuesto diversos métodos para desarrollar técnicas de manipulación de objetos en la escala micro y nanométrica \cite{ashkin1970acceleration, ashkin1987optical, Ashkin, custance2009atomic, dholakia2011shaping, marago2013optical, romo2020controlled}. Las pinzas ópticas han demostrado ser de utilidad para atrapar y mover microobjetos, incluidos virus y bacterias, teniendo un impacto para el desarrollo tecnológico y médico \cite{ashkin1970acceleration, ashkin1987optical, Ashkin}.  

En 2004, Javier García de Abajo publicó un trabajo donde aborda la posibilidad de manipular nanoobjetos mediante microscopios electrónicos de transmisión (TEMs por sus siglas en inglés) \cite{GarciadeAbajo0}. Posteriormente se observó experimentalmente que los TEM pueden ser usados para inducir movimiento y hacer rotar a nanopartículas (NPs) \cite{Batson01, zheng2012electron}, lo cual ha motivado al desarrollo de una nueva técnica de manipulación llamada <<pinzas electrónicas>> \cite{Batson, oleshko2005electron, Oleshko}, en analogía con las pinzas ópticas.