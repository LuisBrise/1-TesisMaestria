%--------------------------------------------------------------
%						      Tesis de doctorado
%
% Transferencia de momento angular de electrones rápidos a nanopartículas
%-------------------------------------------------------------
% 					Autor:, Jorge Luis Briseño Gómez   -												   -
%-----------------------------------------------------------------------
\documentclass[11pt,twoside]{book}
%\documentclass[12pt,twoside]{book}
%\documentclass[11pt,twoside,openany]{book}
%-----------------------------------------------------------------------
%       					   PAQUETES 						       -
%-----------------------------------------------------------------------
\usepackage[activeacute,spanish,es-nodecimaldot]{babel}
\usepackage[utf8]{inputenc}
\usepackage{amscd}
\usepackage{amsmath}
\usepackage{fancyhdr}
\usepackage[letterpaper,top=2cm,bottom=2cm,left=2cm,right=2cm]{geometry} %Márgenes
%\usepackage[letterpaper,top=2.66cm,bottom=2.66cm,left=3.0cm,right=3.0cm]{geometry} %Márgenes
%\usepackage[letterpaper,top=2.5cm,bottom=2.5cm,left=2.5cm,right=2.5cm]{geometry} %Márgenes
\usepackage[numbers,sort&compress]{natbib}%bibliography in bibtex
\usepackage{hypernat} %para hacer compatible mi bibliografía natbib con el backref (citar de regreso a las páginas)
%\usepackage[fixlanguage]{babelbib}
%\selectbiblanguage{spanish}
%************************************************************************
% 							Fancy chapters 								*
%************************************************************************
% siempre debe ir despu�s de geometry para que respete los m�rgenes     *
%************************************************************************
\usepackage[JACR2]{fncychap}
%\usepackage[Bjornstrup]{fncychap}
%\usepackage[Sonny]{fncychap}
%\usepackage[Lenny]{fncychap}
%\usepackage[PetersLenny]{fncychap}
%%\usepackage[Conny]{fncychap} %This does NOT work
%%\usepackage[Glenn]{fncychap} %This does NOT work
%%\usepackage[Rejne]{fncychap} %This does NOT work
%%\usepackage[Bjarne]{fncychap} %This does NOT work
\ChTitleVar{\raggedleft\huge\sffamily\bfseries} %da formato al t\'itulo del cap\'itulo: \raggedleft=pegado a la derecha, \sffamily=Sans Serif (helv\'etica), \bfseries= en enegritas
%%%%%%%%%%%%%%%%%%%%%%%%%%%%%%%%%%%%%%%%%%%%%%%%%%%%%%%%%%
%\usepackage{amsthm}
\usepackage{amsfonts}
\usepackage{mathrsfs}
\usepackage{eucal}
\usepackage{amssymb}
\usepackage{graphics}
\usepackage{graphicx} %Loading the package
\graphicspath{{17-imagenes/2-TeoriaMetodos}} %Setting the graphicspath

%____________________________
\usepackage{caption}
\usepackage{subcaption}
%____________________________
\usepackage{epsfig}
\usepackage{xspace}
\usepackage{latexsym}
%\usepackage[spanish]{babel} %para usar español. Settings:
%
%
%----------------------------------------------
\addto\captionsspanish{%
	\def\bibname{Referencias}%
	\def\tablename{Tabla}%
}
%
%
%----------------------------------------------
%----------------------------------------------
%----------------------------------------------
\usepackage{makeidx}

% Paquetes para texto de relleno
 
% Blindtext
% Opciones pangram, bible, random (defecto)
\usepackage[pangram]{blindtext}

%------------------------------------------------
%%%%%%%espacios para portada UNAM  %%%%%%%
\usepackage{setspace}
%%%%%%%%%%%%%%%% Attractive boxed equations %%%%%%%%%
\usepackage{empheq}
\usepackage[most]{tcolorbox}
\newtcbox{\mymath}[1][]{%
	nobeforeafter, math upper, tcbox raise base,
	enhanced, colframe=orange!90!white,
	colback=orange!2!white, boxrule=1.5pt,
	#1}
%%%%%%%%%%%%%%%% Otra cajita bonita  %%%%%%%%%
\usepackage{tcolorbox}
%\newtcolorbox{mybox}[1]{nobeforeafter, colframe=orange!94!black, colback=orange!2!white,fonttitle=\bfseries,title=#1}
%\newtcolorbox{mybox2}[1]{nobeforeafter, colframe=blue!94!black, colback=blue!2!white,fonttitle=\bfseries,title=#1}
\newtcolorbox{mybox}[1]{ colframe=orange!94!black, colback=orange!2!white,fonttitle=\bfseries,title=#1}
\newtcolorbox{mybox2}[1]{ colframe=blue!94!black, colback=blue!2!white,fonttitle=\bfseries,title=#1}
%para usar:
%\begin{mybox}{titulo}
%	contenido
%\end{mybox}{Ejemplo 5.1: Movimiento del Ciclotrón}	
%%%%%%%%%%%%%%%% Ecuaciones, \'indice y figuras con v\'inculos %%%%%%%%%%%%%%%%%%%%%%%%%%%%%%%%%
%
%
%
%\usepackage[pdftex,linktoc=all]{hyperref}% (*) linktoc=all hace que en el índice tanto los títulos como los números de páginas tengan links; (**) backref=page indica en la bibliografia desde qué página se hizo una cita
\usepackage[pdftex,linktoc=all,backref=page]{hyperref}% (*) linktoc=all hace que en el índice tanto los títulos como los números de páginas tengan links; (**) backref=page indica en la bibliografia desde qué página se hizo una cita
\hypersetup{
	colorlinks=true,
	linkcolor=orange,
	citecolor=orange,
	filecolor=orange,      
	urlcolor=orange,
}
%
% esto que viene es para que en las referencias aparezcan en español las leyendas 
\renewcommand*{\backref}[1]{}
\renewcommand*{\backrefalt}[4]{%
	\ifcase #1%
	\or [citado en la pág.~#2.]%
	\else [citado en las págs.~#2.]%
	\fi%
}
\renewcommand*{\backrefsep}{, }
\renewcommand*{\backreftwosep}{ y~}
\renewcommand*{\backreflastsep}{ y~}
% Explain list of backreferences.
% https://tex.stackexchange.com/a/70953/1340
%\renewcommand{\bibpreamble}{%
%	[Frente a cada referencia se indican las páginas en las que ha sido citada.]%
%	\par\bigskip}
%
%
%
%%%%%%%%%%%%%%%% Figura bla en negritas %%%%%%%%%%%%%%%%%%%%%%%%%%%%%%%%%
\usepackage[labelfont=bf,font=footnotesize]{caption}
%%%%%%%%%%%%%%%% Figuras donde quiera %%%%%%%%%%%%%%%%%%%%%%%%%%%%%%%%%
\usepackage{float}
%%%%%%%%%%%%%%%% Texto de colores %%%%%%%%%%%%%%%%%%%%%%%%%%%%%%%%%
\usepackage{xcolor}
%%%%%% simbolos arriba y abajo de igual %%%%%%%%%%%%%%%%%%%%%%%%%%%%%%%%%
\usepackage{stackengine}
\newcommand\stackequal[2]{%
	\mathrel{\stackunder[2pt]{\stackon[4pt]{=}{$\scriptscriptstyle#1$}}{%
			$\scriptscriptstyle#2$}}}
%%%%%% cancelar terminos %%%%%%%%%%%%%%%%%%%%%%%%%%%%%%%%%		
\usepackage[makeroom]{cancel}
%%%%%% otras cursivas %%%%%%%%%%%%%%%%%%%%%%%%%%%%%%%%%
\usepackage{mathrsfs}
%%%%%%%%%%%%%%%%%%%%% Fancy Headers %%%%%%%%%%%%%%%%%%%%%%
\pagestyle{fancy}

\newcommand{\lineaizq}[1]{#1\hspace{-\textwidth}\protect\rule[-2mm]{\textwidth}{0.2mm}}
\newcommand{\lineader}[1]{\protect\rule[-2mm]{\textwidth}{0.2mm}\hspace{-\textwidth}#1}

\newcommand{\capitulo}[3]{\chapter{#1}
                        	\markboth{\lineaizq{\sc Cap\'itulo \thechapter}}
                        	{\lineader{\sc #3}}
                         	\input{#2}\newpage\thispagestyle{plain}}
              
\newcommand{\sincapitulo}[2]{\chapter*{#1}\addcontentsline{toc}{chapter}{#1}
                           \markboth{\lineaizq{\sc #1}}{\lineader{\sc #1}}
                           \input{#2}\newpage\thispagestyle{plain}}       
                       
                                                    
\newcommand{\apendice}[2]{\chapter*{#1}\addcontentsline{toc}{chapter}{#1}
                           \markboth{\lineaizq{\sc #1}}{\lineader{\sc #1}}
                           \input{#2}\newpage\thispagestyle{plain}}
                                           
                                                                                                        
%---------------------- Integrales -------------------------------------
\newcommand{\vint}[1]{\int_V {#1}\,d^3r}
\newcommand{\sint}[1]{\oint_S {#1}\cdot d\vv{a}}
\newcommand{\tint}[1]{\int_{-\infty}^{\infty} {#1}\,dt}
%---------------------- Transformadas de Fourier -------------------------------------
\newcommand{\fou}[1]{\tint{#1 e^{i\omega t}}}
\newcommand{\ifou}[2]{\int_{-\infty}^{\infty} {#1} e^{-i #2 t}\,\frac{d #2}{2\pi}}
%---------------------- Derivadas -------------------------------------
\newcommand{\dt}[1]{\frac{d}{dt} {#1}}
%---------------------- Figuras ----------------------------------------
%\newcommand{\fref}[1]{Fig.(\ref{#1})}
%-----------------------------------------------------------------------
%--------------- My symbols ------------------
\DeclareMathAlphabet\mathbfcal{OMS}{cmsy}{b}{n}
%--------------- Tensors ------------------
\newcommand{\tensa}[1]{\overset{\lower.5em\hbox{\text{\tiny$\leftrightarrow$}}}{\mathbf{#1}}}
\newcommand{\tensE}[1]{\overset{\lower.5em\hbox{\text{\hspace{0.8 em}\tiny$\leftrightarrow$ {\hspace{-0.2em}\scriptsize E}}}}{\mathbf{#1}}}
\newcommand{\tensH}[1]{\overset{\lower.5em\hbox{\text{\hspace{0.8 em}\tiny$\leftrightarrow$ {\hspace{-0.2em}\scriptsize H}}}}{\mathbf{#1}}}
\newcommand{\tensb}[1]{\overset{\lower.5em\hbox{\text{\tiny$\leftrightarrow$}}}{\pmb{#1}}}
%-------------- Vectors -----------------
\newcommand{\vv}[1]{\vec{\mathbf{#1}}}
\newcommand{\vvb}[1]{\vec{\pmb{#1}}}
%---------------- Others -----------------
\newcommand{\var}[1]{\left( #1 \right) }
\newcommand{\vars}[0]{\left( \vv{r}, t \right) }
\newcommand{\varsw}[0]{\left( \vv{r}, \omega \right) }
\newcommand{\varsk}[0]{\left( \vv{k}, \omega \right) }
\newcommand{\Var}[1]{\left[ #1 \right] }
\newcommand{\abs}[1]{\left| #1 \right| }
\newcommand{\w}[0]{\omega }
\newcommand{\EE}[1]{\vec{\mathbf{E}}_{\text{#1}}}
\newcommand{\BB}[1]{\vec{\mathbf{B}}_{\text{#1}}}
\newcommand{\HH}[1]{\vec{\mathbf{H}}_{\text{#1}}}
\newcommand{\kk}[1]{\kappa_{#1}}
\newcommand{\ee}[1] {\hat{\mathbf{e}}_{#1}}
\newcommand{\wbgv}[0]{ \left( \frac{\left|\omega\right|b}{\gamma v} \right) }
\newcommand{\epsr}[0]{ \tilde{\epsilon} }
\newcommand{\intt}[1]{\int_{-\infty}^{\infty} #1 \, dt}
\newcommand{\intw}[1]{\int_{-\infty}^{\infty} #1 \, d\omega}
\newcommand{\intwo}[1]{\int_{0}^{\infty} #1 \, d\omega}
\newcommand{\intS}[1]{\oint_S #1\cdot d\vv{S}}
\newcommand{\intV}[1]{\int_V #1 dV}
\newcommand{\intTE}[1]{\int_{\text{T.E.}} #1 \, d^3 r}
\newcommand{\intTER}[1]{\int_{\text{T.E.R.}} #1 \, d^3 k}
\newcommand{\re}[1]{{\rm Re}\Var{ #1 }}
\newcommand{\im}[1]{{\rm Im}\Var{ #1 }}
\newcommand{\rmi}[0]{{\rm i}}
\newcommand{\rme}[0]{{\rm e}}
\newcommand{\wR}[0]{\var{\frac{|\w|R}{v\gamma}}}
\newcommand{\comentar}[1]{}

%%%%%%%%%%%%%%%%%%%%%   Text colors  %%%%%%%%%%%%%%%%%%%%%
\newcommand{\cgs}[1]{{\leavevmode\color{magenta}\left(#1\right)}}
%%%%%%%%%%%%%%%%%%%%%%  Page layout  %%%%%%%%%%%%%%%%%%%%%%

\pagestyle{myheadings}

\vfuzz2pt % Don't report over-full v-boxes if over-edge is small
\hfuzz2pt % Don't report over-full h-boxes if over-edge is small
\setlength{\parskip}{1mm}

\renewcommand{\baselinestretch}{1}
%%%%%%%%%%%%%%%%%%%%%    TÍtulo y autor   %%%%%%%%%%%%%%%%%%%%%%

\date{2023}
\title{ Transferencia de momento angular de electrones rápidos a nanopartículas}
\author{Jorge Luis Briseño Gómez}

%%%%%%%%%%%%%%%%%%%%%    Documento     %%%%%%%%%%%%%%%%%%%%%%
\makeindex
%%--------------------------"�"�$"�$"�"%"$�%"�&%"&%$
%%%%%%%%%%%%%%%%%%%%%    espaciamiento de p�rrafos     %%%%%%%%%%%%%%%%%%%%%%
%\setlength{\parindent}{2em}  %sangr�a
%\setlength{\parskip}{1.1em}   %espacio entre p�rrafos
\renewcommand{\baselinestretch}{1.16}   %espacio entre l�neas
%%%%%%%%%%%%%%%%%%%%%%%%%%%%%%%%%%%%%%%%%%%%%%%%%%%%%%%%%%%%%
%---------para solo compilar lo que me interesa-----------
\usepackage{comment}
%\excludecomment{toexclude} % you can name the comment as you wish
\includecomment{toexclude} %cambiar el anterior a este para incluir de nuevo
%
% tomado de https://tex.stackexchange.com/questions/7052/how-do-i-choose-which-sections-to-compile-from-a-latex-document
%
\usepackage[nottoc]{tocbibind} % para añadir la bibliografía al índice
%\selectlanguage{spanish}
%%%%%%%%%%%%%%%%%%%%%%%%%%%%%%%%%%%%%%%%%%%%%%%%%%%%%%%%%%%%%%%%%%%%%%%%%%%%%%%%
%%%%%%%%%%%%%%%%%%%%%%%%%%%%%%%%%%%%%%%%%%%%%%%%%%%%%%%%%%%%%%%%%%%%%%%%%%%%%%%%
%%%%%%%%%%%%%%%%%%%%%%%%%%%%%%%%%%%% Tabla de Alejandro %%%%%%%%%%%%%%%%%%%%%%%%%%%%%%%%%%%%%%%%%%%%%%%%%%%%%%%%%%%%%%%%%%%%%%%%%%%%%%%%
\usepackage{booktabs}       % professional-quality tables
\usepackage{xcolor} %este y el de abajo es para definir colores y ponerlos en la tabla, respectivamente
\usepackage{color, colortbl}
\usepackage{tabularx} % for 'tabularx' environment
\usepackage{makecell}
%
\usepackage{tikz}
\usepackage{bbding}
\usepackage{pifont}
\usepackage{wasysym}
%%%%%%%%%%%%%%%%%%%%%%%%%%%%%%%%%%%%%%%%%%%%%%%%%%%%%%%%%%%%%%%%%%%%%%%%%%%%%%%%

\begin{document}
%%-------------------------------------------------------------------------------------
%--------------------------------- START EXCLUIMOS--------------------------------------
\begin{toexclude}
%%-------------------------------------------------------------------------------------
%
%%%%%%%%%%%%%%%%%%%%%  Portada Oficial UNAM %%%%%%%%%%%%%%%%%%%%%%
\begin{titlepage}
	\begin{center}
		
		\begin{figure}
			\centering
			\includegraphics[scale=.4]{17-imagenes/Unam.png}
		\end{figure}
		
		\begin{spacing}{1.5}
			\textbf{\large UNIVERSIDAD NACIONAL AUT\'ONOMA DE M\'EXICO}\\[-0.2cm]
			{\large POSGRADO EN CIENCIAS F\'ISICAS}
		\end{spacing}
		
		\vfill
		
		\begin{spacing}{1.5}
			\textbf{\large TRANSFERENCIA DE MOMENTO ANGULAR DE ELECTRONES R\'APIDOS A NANOPART\'ICULAS}
		\end{spacing}
		\vfill
		
		\begin{spacing}{1.2}
		    \textbf{\large TESIS}\\[0.2cm]
			{\large QUE PARA OPTAR POR EL GRADO DE:}\\ 
			\textbf{\large MAESTRO EN CIENCIAS (F\'ISICA)}\\
			\vspace*{\fill}
			{\large PRESENTA:}\\[0.2cm]
			\textbf{\large JORGE LUIS BRISEÑO GÓMEZ}
			\vspace*{\fill}
			
			\textbf{\large TUTOR:}\\[0.2cm]
			{\large DR. ALEJANDRO REYES CORONADO}\\
			{\small FACULTAD DE CIENCIAS, UNAM}\\[0.5cm]
			\vspace*{\fill}
			
			\textbf{\large MIEMBROS DEL COMIT\'E TUTOR:}\\[0.2cm]
			{\large DR. RUB\'EN GERARDO BARRERA Y P\'EREZ}\\
			{\small INSTITUTO DE F\'ISICA, UNAM}\\
			{\large DR. RA\'UL PATRICIO ESQUIVEL SIRVENT}\\
			{\small INSTITUTO DE F\'ISICA, UNAM}\\[0.5cm]
		\end{spacing}
		
		{\large CIUDAD DE M\'EXICO, JULIO DE 2023}
		
	
	\end{center}
\end{titlepage}
%%%%%%%%%%%%%%%%%%% p�gina en blanco %%%%%%%%%%%%%%%%%%%%%%%%
%\clearpage\null\newpage
%%%%%%%%%%%%%%%%%%%%%%%%%%%%%%%%%%%%%%%%%%%%%%%%%%%%%%%%
%
%
%
%--------------------- Dedicatoria --------------------- 
%
\begin{flushright}
\null\vspace{\stretch{1}}
\Large{\textit{A Any.}}\\
\Large{\textit{A mi madre, a Tita y a Robin.}}\\
\vspace{3cm}
\large {\guillemotleft\textit{But still try, for who knows what is possible?}\guillemotright }\\
\normalsize{Michael Faraday.}\\
\vspace{1cm}
\large {\guillemotleft\textit{De ilusiones así va uno viviendo.}\guillemotright }\\
\normalsize{Julio Cortázar.}
%\normalsize{De una carta de Julio Cortázar a Paco Porrúa, 22 de mayo de 1961.}
\vspace{\stretch{3}}\null
\end{flushright}

%
%
%---------------------  Indice --------------------- 
%
{
	\hypersetup{linkcolor=black}
	\tableofcontents             %�ndice general
}
\setlength{\parskip}{1mm}

%--------------------- Agradecimientos  --------------------- 
%
\sincapitulo{Agradecimientos}{0-0-Agradecimientos.tex}
%
%%%%%%%%%%%%%%%%%%%%%%%%%%%%%%%%%%%%%%%%%%%%%%%%%%%%%%%
%
%%%%%%%%%%%%%%%%%%%%%%%%%%%%%%%%%%%%%%%%%%%%%%%%%%%%%%%%%
%\frontmatter (antes de empezar pone numeros romanos)
%\mainmatter (empieza a poner numeros arabigos)
%
%-------------------------------------------------------
%
%
%--------------------- CUERPO ------------------------
%
\sincapitulo{Resumen}{0-1-Resumen.tex}
\sincapitulo{Abstract}{0-2-Abstract.tex}
\capitulo{Introducción}{1-Introduccion.tex}{Introducción}
\capitulo{Teoría y métodos}{2-Teoria-y-metodos.tex}{Teor\'ia y métodos}
%\capitulo{Transferencia de momento angular a nanopartículas pequeñas}{Nanoparticulas-pequenas.tex}{Transferencia de momento angular a nanopartículas pequeñas}
%
%-------------------------------------------------------
%%-------------------------------------------------------------------------------------
%--------------------------------- END EXCLUIMOS--------------------------------------
\end{toexclude}
%
\capitulo{Resultados y discusión}{ResultadosyDiscusion.tex}{Resultados y discusión}
\sincapitulo{Conclusiones y perspectivas}{Conclusionesyperspectivas.tex}
%%-------------------------------------------------------------------------------------
%%-------------------------------------------------------------------------------------
%--------------------------------- START EXCLUIMOS--------------------------------------
\begin{toexclude}
%%-------------------------------------------------------------------------------------
%
%-------------------------------------------------------
%
%-----------------    Apéndice A ------------------------
%
\setcounter{equation}{0}
\setcounter{figure}{0}
\renewcommand{\theequation}{A.\arabic{equation}}
\renewcommand{\thefigure}{A.\arabic{figure}}
\apendice{Apéndice A\\ Solución de Mie extendida}{A-Apendice-Mie.tex}
%
%------------------------------------------------------------------------------------
%
%------------------------------------------------------------------------------------
%-----------------    Apéndice B ------------------------
%
\setcounter{equation}{0}
\setcounter{figure}{0}
\renewcommand{\theequation}{B.\arabic{equation}}
\renewcommand{\thefigure}{B.\arabic{figure}}
%\apendice{Apéndice B\\  Solución de Mie extendida}{Apendice-Mie.tex} 
%
%------------------------------------------------------------------------------------
%
%%-------------------------------------------------------------------------------------
%
%-----------------    Apéndice C ------------------------
%
\setcounter{equation}{0}
\setcounter{figure}{0}
\renewcommand{\theequation}{C.\arabic{equation}}
\renewcommand{\thefigure}{C.\arabic{figure}}
%\apendice{Apéndice C\\ Efecto de las correcciones por tamaño a la función dieléctrica en la TMA}{Apendice-size-corrections.tex}
%\apendice{Apéndice C\\ Efecto de las correcciones por tamaño a $\varepsilon\var{\w}$ en la TMA}{Apendice-size-corrections.tex}
%
%------------------------------------------------------------------------------------
%-------------------------------------------------------
%
%-----------------    Apéndice D ------------------------
%
\setcounter{equation}{0}
\setcounter{figure}{0}
\renewcommand{\theequation}{D.\arabic{equation}}
\renewcommand{\thefigure}{D.\arabic{figure}}
%\apendice{Apéndice D\\Diferencias con un método previo para el límite de partícula pequeña}{Apendice-GdA.tex}
%
%------------------------------------------------------------------------------------
%------------------------------------------------------
%
%-----------------    Apéndice E ------------------------
%
\setcounter{equation}{0}
\setcounter{figure}{0}
\renewcommand{\theequation}{E.\arabic{equation}}
\renewcommand{\thefigure}{E.\arabic{figure}}
%\apendice{Apéndice E\\ Convergencia de los cálculos de la TMA}{Apendice-convergencia.tex}
%
%------------------------------------------------------------------------------------
%------------------------------------------------------
%
%------------------------------------------------------------------------------------
%-------------------------------------------------------
%
%-----------------    Apéndice F ------------------------
%
\setcounter{equation}{0}
\setcounter{figure}{0}
\renewcommand{\theequation}{F.\arabic{equation}}
\renewcommand{\thefigure}{F.\arabic{figure}}
%\apendice{Apéndice F\\ Tiempos de cómputo de la TMA}{Apendice-tiempos-de-computo.tex}
%
%------------------------------------------------------------------------------------
%------------------------------------------------------
%
%------------------------------------------------------------------------------------
%-------------------------------------------------------
%------------------------------------------------------------------------------------
%-------------------------------------------------------
%
%-----------------    Lista de publicaciones------------------------
%
%\sincapitulo{Lista de publicaciones}{Publicaciones.tex}
%------------------------------------------------------------------------------------
%------------------------------------------------------
%
%------------------------------------------------------------------------------------
%-------------------------------------------------------
%------------------------------------------------------------------------------------

%
%%-------------------------------------------------------------------------------------
%--------------------------------- END EXCLUIMOS--------------------------------------
\end{toexclude}
%%-------------------------------------------------------------------------------------
%
%-----bibtex references----------------
%\bibliographystyle{elsarticle-num-names}
%\bibliographystyle{elsarticle-num-names-JACR}
%\bibliographystyle{plainnat} 
%\bibliographystyle{unsrt} 
\bibliographystyle{unsrt-jacr} 
%
%\bibliographystyle{babunsrt-fl} 
%
%babplain, babunsrt,
%bababbrv, and babalpha babamspl
%bababbr3 and babplai3
%gerabbrv, geralpha, gerapali, gerplain,
%gerunsrt
%
%\bibliographystyle{abbrv} 
\clearpage
\markboth{Referencias}{Referencias}
%\bibliography{references}
\bibliography{references-links-arc}
%-----------------------------------
\end{document}