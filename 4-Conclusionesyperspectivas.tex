% !TeX root = ./0-Tesis-de-maestria-JLBG.tex

\label{conclusiones}

\section{Conclusiones}

En esta tesis se ha presentado un análisis teórico detallado de la transferencia de momento angular que tiene lugar entre un haz de electrones y nanopartículas esféricas dentro de un microscopio electrónico de transmisión y de barrido (STEM), utilizando un enfoque de electrodinámica clásica. Se ha desarrollado una metodología que permite estudiar la transferencia de momento angular con alta eficiencia computacional. Se han obtenido expresiones semi-analíticas exactas y cerradas de la densidad espectral de transferencia de momento angular, modelando la respuesta electromagnética de la nanopartícula mediante su función dieléctrica $\epsilon(\omega)$.

Para calcular la transferencia de momento angular $\Delta \vv{L}$, es necesario integrar la densidad espectral $\vv{\mathcal{L}t}$ en todo el espacio de frecuencias. La metodología presentada en esta tesis permite estudiar la transferencia de momento angular a nanopartículas de cualquier tamaño en la nanoescala, incluyendo el caso de nanopartículas grandes de hasta 50 nm de radio. Hasta donde tiene conocimiento el autor, no se han realizado cálculos de esta naturaleza con métodos numéricos debido a las limitaciones que provienen de los tiempos de cómputo, en particular, para nanopartículas grandes, de más de $10$ nm de radio, caracterizadas por funciones dieléctricas realistas que no se basen en el modelo de Drude.

Además, se ha implementado un código en lenguaje C que calcula la contribución del campo electromagnético externo del electrón a la transferencia de momento angular. Se realizaron los cálculos de $\Delta \vv{L}$ en función del parámetro de impacto, de la velocidad del electrón y del tamaño de la superficie de integración. Al obtener siempre resultados que se anulan, se ha ganado confianza en la validez de las soluciones semi-analíticas, ya que los términos del tensor de esfuerzos de Maxwell que contienen únicamente al campo electromagnético del electrón $\vv{E}_{\rm ext}$ no contribuyen a la transferencia de momento angular total. 

Con las expresiones de las integrales que contienen también a los campos electromagnéticos esparcidos por las nanopartículas, se puede utilizar el presente trabajo para explorar la dinámica angular en la interacción de haces de electrones con nanopartículas en un amplio rango de tamaños y materiales. En conclusión, el presente trabajo proporciona una herramienta importante para el estudio teórico de la transferencia de momento angular dentro de un STEM, lo que puede tener importantes aplicaciones en campos como la nanotecnología y la medicina, con  el desarrollo de las pinzas electrónicas.

\section{Trabajo a futuro}

En esta tesis se ha investigado la interacción entre el campo electromagnético producido por un electrón rápido y una nanopartícula, aunque aún quedan varios temas por explorar. Con el fin de profundizar en este tema, se pretende extender el código en lenguaje C para calcular tanto los campos electromagnéticos externos del electrón como los esparcidos por la nanopartícula, lo que permitirá programar las integrales necesarias para obtener la transferencia de momento angular total. Esta metodología permite separar la contribución eléctrica de la magnética y estudiar la contribución de distintos órdenes multipolares a la transferencia de momento angular.

En el futuro, se planea utilizar esta metodología para calcular la transferencia de momento angular de electrones rápidos a nanopartículas de diferentes tamaños y materiales (dieléctricos y plasmónicos) en toda la escala nano. Para ello, solo será necesario conocer la función dieléctrica causal del material de la nanopartícula. Una vez obtenidos los resultados, se deben comparar con los publicados previamente para nanopartículas pequeñas, de hasta 5 nm de radio, para corroborar la validez de la solución semi-analítica o corregirla si es necesario. Posteriormente, se podrán extender los resultados para nanopartículas más grandes de hasta 50 nm de radio.

Por último, se propone explorar el comportamiento de la interacción utilizando funciones dieléctricas espacialmente locales y no locales. Esto se debe a que experimentalmente se ha observado una interacción atractiva que aún no ha podido explicarse con los modelos publicados hasta el momento. Es probable que esto se deba a que, cuando el parámetro de impacto es lo suficientemente pequeño, la naturaleza no local de las funciones dieléctricas tenga algún papel importante.