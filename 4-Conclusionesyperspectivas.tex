% !TeX root = ./0-Tesis-de-maestria-JLBG.tex

\label{conclusiones}

\section{Conclusiones}

En esta tesis se ha presentado un análisis teórico detallado de la transferencia de momento angular (TMA) de un haz de electrones a una nanopartícula (NP) esférica dentro de un microscopio electrónico de transmisión y de barrido (STEM), utilizando un enfoque de electrodinámica clásica. Se ha desarrollado una metodología que permite estudiar la transferencia de momento angular con alta eficiencia computacional. Se han obtenido expresiones semi-analíticas exactas y cerradas de la densidad espectral de transferencia de momento angular, modelando la respuesta electromagnética de la nanopartícula mediante su función dieléctrica $\epsilon(\omega)$.

Para calcular la transferencia de momento angular $\Delta \vv{L}$, es necesario integrar la densidad espectral $\vv{\mathcal{L}}$ en todo el espacio de frecuencias. Anteriormente, empleando otra metodología de cálculo numérico basado en cubaturas, solamente se ha podido estudiar la TMA para NPs de hasta 10 de radio para NPs modeladas a partir de una función dieléctrica dada por el modelo de Drude, o bien de hasta 5 nm de radio para NPs hechas de materiales plasmónicos como oro, bismuto y plata. Se estima que la metodología presentada en esta tesis permitirá estudiar la transferencia de momento angular a nanopartículas de cualquier tamaño en la nanoescala, incluyendo el caso de nanopartículas grandes de hasta 50 nm de radio. 

También se ha implementado un código en lenguaje C que calcula la contribución del campo electromagnético externo del electrón a la transferencia de momento angular (la cual es cero), a modo de prueba de las expresiones obtenidas. Se realizaron los cálculos de $\Delta \vv{L}$ en función del parámetro de impacto, de la rapidez del electrón y del radio de la superficie esférica de integración. Al obtener siempre resultados que se anulan, se ha ganado confianza en las soluciones semi-analíticas.

Con las expresiones de las integrales que contienen también a los campos electromagnéticos esparcidos por las nanopartículas, se puede utilizar el presente trabajo para explorar la dinámica angular en la interacción de haces de electrones con nanopartículas en un amplio rango de tamaños y materiales. En conclusión, el presente trabajo proporciona una herramienta poderosa para el estudio teórico de la transferencia de momento angular dentro de un STEM, lo que puede tener importantes aplicaciones en campos como la nanotecnología y la medicina, con  el desarrollo de las pinzas electrónicas.

\section{Trabajo a futuro}

En esta tesis se ha llevado a cabo una investigación teórica sobre la interacción entre el campo electromagnético generado por un electrón rápido y una nanopartícula. Como resultado de esta investigación, se han obtenido expresiones semi-analíticas que permitirán el cálculo eficiente de la TMA del  haz de electrones a la NP. Estas expresiones podrán utilizarse para el cálculo de la TMA en nanopartículas grandes, de hasta 50 nm de radio, compuestas por materiales plasmónicos y dieléctricos. Se pretende ampliar el código en lenguaje C para calcular tanto los campos electromagnéticos externos generados por el electrón como los campos esparcidos por la nanopartícula. Esto permitirá programar las integrales necesarias para obtener la TMA total, que incluye distintas contribuciones, como la interacción ext-ext (producida solo por el campo electromagnético del electrón), la interacción electrón-NP y la reacción de radiación (originada solo por el campo esparcido por la nanopartícula).

En el futuro, se planea utilizar esta metodología para calcular la transferencia de momento angular de electrones rápidos a nanopartículas de diferentes tamaños y materiales (dieléctricos y plasmónicos) en toda la escala nano. Para ello, solo será necesario conocer la función dieléctrica causal que caracterice la respuesta electromagnética del material del que está hecha la nanopartícula. Para poner a prueba el nuevo código, se pueden comparar los resultados con los obtenidos previamente para el caso de NPs pequeñas, de hasta 5 nm de radio, para diferentes materiales. Posteriormente, se podrán extender los resultados para nanopartículas más grandes de hasta 50 nm de radio, analizando la TMA en función de los parámetros relevantes del problema: parámetro de impacto y rapidez de los electrones..

Por último, se propone explorar el comportamiento de la TMA utilizando funciones dieléctricas espacialmente no locales. Esto se debe a que experimentalmente se ha observado una interacción atractiva que aún no ha podido explicarse con los modelos con los que se cuentan hasta el momento. Es probable que esto se deba a que, cuando el parámetro de impacto es lo suficientemente pequeño, la naturaleza no local de las funciones dieléctricas tenga un papel importante.