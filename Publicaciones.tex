\hypertarget{listadepublicaciones}{En} el transcurso del doctorado participé en la publicación de los siguientes trabajos arbitrados:
\vspace{0.5cm}

%
%
\begin{mybox2}{\centering 1.- Electronic tweezers for magnesium oxide nanoparticles (2019)}
	\textbf{Jos\'e \'Angel Castellanos-Reyes}, Jes\'us Castrej\'on-Figueroa, Carlos Maciel-Escudero, Alejandro Reyes-Coronado,
	\href{https://doi.org/10.1016/j.matpr.2019.03.163}{Materials Today: Proceedings,
		Volume 13,
		Part 2, 
		341-348 
		(2019).}
\end{mybox2}{}
%
En este trabajo se presenta un estudio teórico sobre la transferencia de momento lineal de haces de electrones a nanopartículas de óxido de magnesio con un enfoque a la aplicabilidad de las pinzas electrónicas para la manipulación de nanopartículas dieléctricas.
\vspace{0.5cm}

%
%
\begin{mybox2}{\centering 2.- Time-dependent forces between a swift electron and a small nanoparticle within the dipole approximation (2021)}
J. Castrejón-Figueroa, \textbf{J. Á. Castellanos-Reyes}, C. Maciel-Escudero, A. Reyes-Coronado, R. G. Barrera,
\href{https://doi.org/10.1103/PhysRevB.103.155413}{Physical Review B, Volume 103, Number 15, 155413 (2021).}
\end{mybox2}{}
%
En este trabajo se presenta un estudio teórico, utilizando la aproximación cuasiestática, sobre el comportamiento temporal de la dinámica lineal de nanopartículas pequeñas debido a su interacción con haces de electrones. 
\vspace{0.5cm}

%
%
\begin{mybox2}{\centering 3.- Angular dynamics of small nanoparticles induced by non-vortex electron beams (2021)}
	\textbf{J. Á. Castellanos-Reyes}, J. Castrejón-Figueroa, A. Reyes-Coronado,
	\href{https://doi.org/10.1016/j.ultramic.2021.113274}{Ultramicroscopy,
		Volume 225,
		113274
		(2021).}
\end{mybox2}{}
%
En este trabajo se presenta un estudio teórico, utilizando el límite de partícula pequeña, sobre el comportamiento temporal de la dinámica angular de nanopartículas pequeñas debido a su interacción con haces de electrones. En este trabajo se encuentra el contenido del \hyperlink{nanoparticulaspequeñas}{Capítulo 3} de esta tesis, así como los resultados del \hyperlink{resultadosydiscusion}{Capítulo 4} correspondientes al límite de partícula pequeña y la discusión del \hyperlink{AppendixGdA}{Apéndice D}.
\vspace{0.5cm}

Además, en este momento se encuentra en proceso de revisión el trabajo

\vspace{0.5cm}
%
%
\begin{mybox2}{\centering 4.- Non-causality and numerical convergence impact on the linear momentum transfer by a swift electron to a metallic nanoparticle}
	J. Castrejón-Figueroa, \textbf{J. Á. Castellanos-Reyes}, A. Reyes-Coronado, 
	\href{https://arxiv.org/abs/2106.04032}{arXiv:2106.04032.}
\end{mybox2}{}
%
En este trabajo se presenta un estudio teórico sobre los efectos de la no causalidad y la falta de convergencia numérica en los cálculos de la transferencia de momento lineal de haces de electrones a nanopartículas.
\vspace{0.5cm}

Finalmente, vale la pena mencionar que actualmente nos encontramos trabajando en la escritura de dos trabajos:
%
\begin{itemize}
	\item El primero, cuyos autores son \textbf{J. Á. Castellanos-Reyes}, J. Castrejón-Figueroa y A. Reyes-Coronado, sobre los resultados de la transferencia de momento angular de electrones rápidos a nanopartículas grandes, el cual contiene los resultados principales (sin contar aquellos del límite cuasiestático), del \hyperlink{resultadosydiscusion}{Capítulo 4} de esta tesis.
	\item El segundo, cuyos autores son J. Castrejón-Figueroa, \textbf{J. Á. Castellanos-Reyes} y A. Reyes-Coronado, sobre los resultados de la transferencia de momento lineal de electrones rápidos a nanopartículas grandes que incluyen resultados de la tesis de doctorado de J. Castrejón-Figueroa.
\end{itemize}
